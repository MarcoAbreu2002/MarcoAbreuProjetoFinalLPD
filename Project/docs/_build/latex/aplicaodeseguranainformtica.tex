%% Generated by Sphinx.
\def\sphinxdocclass{report}
\documentclass[letterpaper,10pt,english]{sphinxmanual}
\ifdefined\pdfpxdimen
   \let\sphinxpxdimen\pdfpxdimen\else\newdimen\sphinxpxdimen
\fi \sphinxpxdimen=.75bp\relax
\ifdefined\pdfimageresolution
    \pdfimageresolution= \numexpr \dimexpr1in\relax/\sphinxpxdimen\relax
\fi
%% let collapsible pdf bookmarks panel have high depth per default
\PassOptionsToPackage{bookmarksdepth=5}{hyperref}

\PassOptionsToPackage{booktabs}{sphinx}
\PassOptionsToPackage{colorrows}{sphinx}

\PassOptionsToPackage{warn}{textcomp}
\usepackage[utf8]{inputenc}
\ifdefined\DeclareUnicodeCharacter
% support both utf8 and utf8x syntaxes
  \ifdefined\DeclareUnicodeCharacterAsOptional
    \def\sphinxDUC#1{\DeclareUnicodeCharacter{"#1}}
  \else
    \let\sphinxDUC\DeclareUnicodeCharacter
  \fi
  \sphinxDUC{00A0}{\nobreakspace}
  \sphinxDUC{2500}{\sphinxunichar{2500}}
  \sphinxDUC{2502}{\sphinxunichar{2502}}
  \sphinxDUC{2514}{\sphinxunichar{2514}}
  \sphinxDUC{251C}{\sphinxunichar{251C}}
  \sphinxDUC{2572}{\textbackslash}
\fi
\usepackage{cmap}
\usepackage[T1]{fontenc}
\usepackage{amsmath,amssymb,amstext}
\usepackage{babel}



\usepackage{tgtermes}
\usepackage{tgheros}
\renewcommand{\ttdefault}{txtt}



\usepackage[Bjarne]{fncychap}
\usepackage{sphinx}

\fvset{fontsize=auto}
\usepackage{geometry}


% Include hyperref last.
\usepackage{hyperref}
% Fix anchor placement for figures with captions.
\usepackage{hypcap}% it must be loaded after hyperref.
% Set up styles of URL: it should be placed after hyperref.
\urlstyle{same}

\addto\captionsenglish{\renewcommand{\contentsname}{Contents:}}

\usepackage{sphinxmessages}
\setcounter{tocdepth}{1}



\title{Aplicação de Segurança Informática}
\date{Feb 27, 2024}
\release{0.0.1}
\author{Marco Abreu}
\newcommand{\sphinxlogo}{\vbox{}}
\renewcommand{\releasename}{Release}
\makeindex
\begin{document}

\ifdefined\shorthandoff
  \ifnum\catcode`\=\string=\active\shorthandoff{=}\fi
  \ifnum\catcode`\"=\active\shorthandoff{"}\fi
\fi

\pagestyle{empty}
\sphinxmaketitle
\pagestyle{plain}
\sphinxtableofcontents
\pagestyle{normal}
\phantomsection\label{\detokenize{index::doc}}


\sphinxstepscope


\chapter{Project}
\label{\detokenize{modules:project}}\label{\detokenize{modules::doc}}
\sphinxstepscope


\section{client1 module}
\label{\detokenize{client1:module-client1}}\label{\detokenize{client1:client1-module}}\label{\detokenize{client1::doc}}\index{module@\spxentry{module}!client1@\spxentry{client1}}\index{client1@\spxentry{client1}!module@\spxentry{module}}\index{Client (class in client1)@\spxentry{Client}\spxextra{class in client1}}

\begin{fulllineitems}
\phantomsection\label{\detokenize{client1:client1.Client}}
\pysigstartsignatures
\pysiglinewithargsret{\sphinxbfcode{\sphinxupquote{class\DUrole{w}{ }}}\sphinxcode{\sphinxupquote{client1.}}\sphinxbfcode{\sphinxupquote{Client}}}{\sphinxparam{\DUrole{n}{host}}\sphinxparamcomma \sphinxparam{\DUrole{n}{port}}}{}
\pysigstopsignatures
\sphinxAtStartPar
Bases: \sphinxcode{\sphinxupquote{object}}
\index{create\_new\_user() (client1.Client method)@\spxentry{create\_new\_user()}\spxextra{client1.Client method}}

\begin{fulllineitems}
\phantomsection\label{\detokenize{client1:client1.Client.create_new_user}}
\pysigstartsignatures
\pysiglinewithargsret{\sphinxbfcode{\sphinxupquote{create\_new\_user}}}{}{}
\pysigstopsignatures
\sphinxAtStartPar
Creates a new user with a new private key pair.

\end{fulllineitems}

\index{download\_messages() (client1.Client method)@\spxentry{download\_messages()}\spxextra{client1.Client method}}

\begin{fulllineitems}
\phantomsection\label{\detokenize{client1:client1.Client.download_messages}}
\pysigstartsignatures
\pysiglinewithargsret{\sphinxbfcode{\sphinxupquote{download\_messages}}}{\sphinxparam{\DUrole{n}{messages\_to\_display}}\sphinxparamcomma \sphinxparam{\DUrole{n}{username}}}{}
\pysigstopsignatures
\sphinxAtStartPar
Downloads and saves messages received from the server to a file.
\begin{quote}\begin{description}
\sphinxlineitem{Parameters}\begin{itemize}
\item {} 
\sphinxAtStartPar
\sphinxstyleliteralstrong{\sphinxupquote{messages\_to\_display}} (\sphinxstyleliteralemphasis{\sphinxupquote{str}}) \textendash{} The messages to be downloaded.

\item {} 
\sphinxAtStartPar
\sphinxstyleliteralstrong{\sphinxupquote{username}} (\sphinxstyleliteralemphasis{\sphinxupquote{str}}) \textendash{} The username of the client.

\end{itemize}

\end{description}\end{quote}

\end{fulllineitems}

\index{load\_existing\_user() (client1.Client method)@\spxentry{load\_existing\_user()}\spxextra{client1.Client method}}

\begin{fulllineitems}
\phantomsection\label{\detokenize{client1:client1.Client.load_existing_user}}
\pysigstartsignatures
\pysiglinewithargsret{\sphinxbfcode{\sphinxupquote{load\_existing\_user}}}{}{}
\pysigstopsignatures
\sphinxAtStartPar
Loads an existing user’s private key from a file.
If the file doesn’t exist or the passphrase is incorrect, prompts to create a new user.

\end{fulllineitems}

\index{receive\_messages() (client1.Client method)@\spxentry{receive\_messages()}\spxextra{client1.Client method}}

\begin{fulllineitems}
\phantomsection\label{\detokenize{client1:client1.Client.receive_messages}}
\pysigstartsignatures
\pysiglinewithargsret{\sphinxbfcode{\sphinxupquote{receive\_messages}}}{}{}
\pysigstopsignatures
\sphinxAtStartPar
Receives and displays messages from the server.

\end{fulllineitems}

\index{receive\_public\_key\_and\_mac\_key() (client1.Client method)@\spxentry{receive\_public\_key\_and\_mac\_key()}\spxextra{client1.Client method}}

\begin{fulllineitems}
\phantomsection\label{\detokenize{client1:client1.Client.receive_public_key_and_mac_key}}
\pysigstartsignatures
\pysiglinewithargsret{\sphinxbfcode{\sphinxupquote{receive\_public\_key\_and\_mac\_key}}}{}{}
\pysigstopsignatures
\sphinxAtStartPar
Receives the server’s public key and the encrypted MAC key from the server.

\end{fulllineitems}

\index{remove\_last\_line() (client1.Client method)@\spxentry{remove\_last\_line()}\spxextra{client1.Client method}}

\begin{fulllineitems}
\phantomsection\label{\detokenize{client1:client1.Client.remove_last_line}}
\pysigstartsignatures
\pysiglinewithargsret{\sphinxbfcode{\sphinxupquote{remove\_last\_line}}}{}{}
\pysigstopsignatures
\sphinxAtStartPar
Removes the last line printed on the console.

\end{fulllineitems}

\index{send\_message() (client1.Client method)@\spxentry{send\_message()}\spxextra{client1.Client method}}

\begin{fulllineitems}
\phantomsection\label{\detokenize{client1:client1.Client.send_message}}
\pysigstartsignatures
\pysiglinewithargsret{\sphinxbfcode{\sphinxupquote{send\_message}}}{}{}
\pysigstopsignatures
\sphinxAtStartPar
Sends messages from the client to the server.

\end{fulllineitems}

\index{start() (client1.Client method)@\spxentry{start()}\spxextra{client1.Client method}}

\begin{fulllineitems}
\phantomsection\label{\detokenize{client1:client1.Client.start}}
\pysigstartsignatures
\pysiglinewithargsret{\sphinxbfcode{\sphinxupquote{start}}}{}{}
\pysigstopsignatures
\sphinxAtStartPar
Starts the client.

\end{fulllineitems}


\end{fulllineitems}

\index{decrypt\_private\_key() (in module client1)@\spxentry{decrypt\_private\_key()}\spxextra{in module client1}}

\begin{fulllineitems}
\phantomsection\label{\detokenize{client1:client1.decrypt_private_key}}
\pysigstartsignatures
\pysiglinewithargsret{\sphinxcode{\sphinxupquote{client1.}}\sphinxbfcode{\sphinxupquote{decrypt\_private\_key}}}{\sphinxparam{\DUrole{n}{encrypted\_key\_data}}\sphinxparamcomma \sphinxparam{\DUrole{n}{passphrase}}}{}
\pysigstopsignatures
\sphinxAtStartPar
Decrypts an encrypted private key using a passphrase.
\begin{quote}\begin{description}
\sphinxlineitem{Parameters}\begin{itemize}
\item {} 
\sphinxAtStartPar
\sphinxstyleliteralstrong{\sphinxupquote{encrypted\_key\_data}} (\sphinxstyleliteralemphasis{\sphinxupquote{bytes}}) \textendash{} The encrypted private key data.

\item {} 
\sphinxAtStartPar
\sphinxstyleliteralstrong{\sphinxupquote{passphrase}} (\sphinxstyleliteralemphasis{\sphinxupquote{str}}) \textendash{} The passphrase to decrypt the private key.

\end{itemize}

\sphinxlineitem{Returns}
\sphinxAtStartPar
The decrypted private key or None if decryption fails.

\sphinxlineitem{Return type}
\sphinxAtStartPar
RSA.RsaKey or None

\end{description}\end{quote}

\end{fulllineitems}

\index{encrypt\_private\_key() (in module client1)@\spxentry{encrypt\_private\_key()}\spxextra{in module client1}}

\begin{fulllineitems}
\phantomsection\label{\detokenize{client1:client1.encrypt_private_key}}
\pysigstartsignatures
\pysiglinewithargsret{\sphinxcode{\sphinxupquote{client1.}}\sphinxbfcode{\sphinxupquote{encrypt\_private\_key}}}{\sphinxparam{\DUrole{n}{private\_key}}\sphinxparamcomma \sphinxparam{\DUrole{n}{passphrase}}}{}
\pysigstopsignatures
\sphinxAtStartPar
Encrypts a private key using a passphrase.
\begin{quote}\begin{description}
\sphinxlineitem{Parameters}\begin{itemize}
\item {} 
\sphinxAtStartPar
\sphinxstyleliteralstrong{\sphinxupquote{private\_key}} (\sphinxstyleliteralemphasis{\sphinxupquote{RSA.RsaKey}}) \textendash{} The private key to encrypt.

\item {} 
\sphinxAtStartPar
\sphinxstyleliteralstrong{\sphinxupquote{passphrase}} (\sphinxstyleliteralemphasis{\sphinxupquote{str}}) \textendash{} The passphrase to encrypt the private key.

\end{itemize}

\sphinxlineitem{Returns}
\sphinxAtStartPar
The encrypted private key.

\sphinxlineitem{Return type}
\sphinxAtStartPar
bytes

\end{description}\end{quote}

\end{fulllineitems}


\sphinxstepscope


\section{doorknocking module}
\label{\detokenize{doorknocking:module-doorknocking}}\label{\detokenize{doorknocking:doorknocking-module}}\label{\detokenize{doorknocking::doc}}\index{module@\spxentry{module}!doorknocking@\spxentry{doorknocking}}\index{doorknocking@\spxentry{doorknocking}!module@\spxentry{module}}\index{establish\_L2TP\_IPSEC\_connection() (in module doorknocking)@\spxentry{establish\_L2TP\_IPSEC\_connection()}\spxextra{in module doorknocking}}

\begin{fulllineitems}
\phantomsection\label{\detokenize{doorknocking:doorknocking.establish_L2TP_IPSEC_connection}}
\pysigstartsignatures
\pysiglinewithargsret{\sphinxcode{\sphinxupquote{doorknocking.}}\sphinxbfcode{\sphinxupquote{establish\_L2TP\_IPSEC\_connection}}}{\sphinxparam{\DUrole{n}{server\_ip}}\sphinxparamcomma \sphinxparam{\DUrole{n}{pre\_shared\_key}}\sphinxparamcomma \sphinxparam{\DUrole{n}{username}}\sphinxparamcomma \sphinxparam{\DUrole{n}{password}}}{}
\pysigstopsignatures
\end{fulllineitems}

\index{establish\_ssh\_connection() (in module doorknocking)@\spxentry{establish\_ssh\_connection()}\spxextra{in module doorknocking}}

\begin{fulllineitems}
\phantomsection\label{\detokenize{doorknocking:doorknocking.establish_ssh_connection}}
\pysigstartsignatures
\pysiglinewithargsret{\sphinxcode{\sphinxupquote{doorknocking.}}\sphinxbfcode{\sphinxupquote{establish\_ssh\_connection}}}{\sphinxparam{\DUrole{n}{hostname}}\sphinxparamcomma \sphinxparam{\DUrole{n}{username}}\sphinxparamcomma \sphinxparam{\DUrole{n}{password}}}{}
\pysigstopsignatures
\end{fulllineitems}

\index{install\_packages() (in module doorknocking)@\spxentry{install\_packages()}\spxextra{in module doorknocking}}

\begin{fulllineitems}
\phantomsection\label{\detokenize{doorknocking:doorknocking.install_packages}}
\pysigstartsignatures
\pysiglinewithargsret{\sphinxcode{\sphinxupquote{doorknocking.}}\sphinxbfcode{\sphinxupquote{install\_packages}}}{}{}
\pysigstopsignatures
\end{fulllineitems}

\index{knock\_sequence() (in module doorknocking)@\spxentry{knock\_sequence()}\spxextra{in module doorknocking}}

\begin{fulllineitems}
\phantomsection\label{\detokenize{doorknocking:doorknocking.knock_sequence}}
\pysigstartsignatures
\pysiglinewithargsret{\sphinxcode{\sphinxupquote{doorknocking.}}\sphinxbfcode{\sphinxupquote{knock\_sequence}}}{\sphinxparam{\DUrole{n}{ip\_address}}\sphinxparamcomma \sphinxparam{\DUrole{n}{sequence}}}{}
\pysigstopsignatures
\sphinxAtStartPar
This method receives a remote ip\_address and a sequence. Then
according to the sequence it creates telnet knocks.

\end{fulllineitems}

\index{main() (in module doorknocking)@\spxentry{main()}\spxextra{in module doorknocking}}

\begin{fulllineitems}
\phantomsection\label{\detokenize{doorknocking:doorknocking.main}}
\pysigstartsignatures
\pysiglinewithargsret{\sphinxcode{\sphinxupquote{doorknocking.}}\sphinxbfcode{\sphinxupquote{main}}}{}{}
\pysigstopsignatures
\end{fulllineitems}


\sphinxstepscope


\section{encryption module}
\label{\detokenize{encryption:module-encryption}}\label{\detokenize{encryption:encryption-module}}\label{\detokenize{encryption::doc}}\index{module@\spxentry{module}!encryption@\spxentry{encryption}}\index{encryption@\spxentry{encryption}!module@\spxentry{module}}\index{decrypt\_rsa() (in module encryption)@\spxentry{decrypt\_rsa()}\spxextra{in module encryption}}

\begin{fulllineitems}
\phantomsection\label{\detokenize{encryption:encryption.decrypt_rsa}}
\pysigstartsignatures
\pysiglinewithargsret{\sphinxcode{\sphinxupquote{encryption.}}\sphinxbfcode{\sphinxupquote{decrypt\_rsa}}}{\sphinxparam{\DUrole{n}{data}}\sphinxparamcomma \sphinxparam{\DUrole{n}{private\_key}}}{}
\pysigstopsignatures
\sphinxAtStartPar
Decrypts RSA encrypted data using a given private key.
\begin{quote}\begin{description}
\sphinxlineitem{Parameters}\begin{itemize}
\item {} 
\sphinxAtStartPar
\sphinxstyleliteralstrong{\sphinxupquote{data}} (\sphinxstyleliteralemphasis{\sphinxupquote{bytes}}) \textendash{} The encrypted data.

\item {} 
\sphinxAtStartPar
\sphinxstyleliteralstrong{\sphinxupquote{private\_key}} (\sphinxstyleliteralemphasis{\sphinxupquote{RSA.RsaKey}}) \textendash{} The RSA private key.

\end{itemize}

\sphinxlineitem{Returns}
\sphinxAtStartPar
The decrypted data.

\sphinxlineitem{Return type}
\sphinxAtStartPar
bytes

\end{description}\end{quote}

\end{fulllineitems}

\index{decrypt\_rsa\_in\_chunks() (in module encryption)@\spxentry{decrypt\_rsa\_in\_chunks()}\spxextra{in module encryption}}

\begin{fulllineitems}
\phantomsection\label{\detokenize{encryption:encryption.decrypt_rsa_in_chunks}}
\pysigstartsignatures
\pysiglinewithargsret{\sphinxcode{\sphinxupquote{encryption.}}\sphinxbfcode{\sphinxupquote{decrypt\_rsa\_in\_chunks}}}{\sphinxparam{\DUrole{n}{ciphertext}}\sphinxparamcomma \sphinxparam{\DUrole{n}{private\_key}}\sphinxparamcomma \sphinxparam{\DUrole{n}{chunk\_size}\DUrole{o}{=}\DUrole{default_value}{256}}}{}
\pysigstopsignatures
\sphinxAtStartPar
Decrypts RSA encrypted data in chunks using a given private key.
\begin{quote}\begin{description}
\sphinxlineitem{Parameters}\begin{itemize}
\item {} 
\sphinxAtStartPar
\sphinxstyleliteralstrong{\sphinxupquote{ciphertext}} (\sphinxstyleliteralemphasis{\sphinxupquote{bytes}}) \textendash{} The encrypted data.

\item {} 
\sphinxAtStartPar
\sphinxstyleliteralstrong{\sphinxupquote{private\_key}} (\sphinxstyleliteralemphasis{\sphinxupquote{RSA.RsaKey}}) \textendash{} The RSA private key.

\item {} 
\sphinxAtStartPar
\sphinxstyleliteralstrong{\sphinxupquote{chunk\_size}} (\sphinxstyleliteralemphasis{\sphinxupquote{int}}) \textendash{} The size of each decryption chunk in bytes.

\end{itemize}

\sphinxlineitem{Returns}
\sphinxAtStartPar
The decrypted data.

\sphinxlineitem{Return type}
\sphinxAtStartPar
bytes

\end{description}\end{quote}

\end{fulllineitems}

\index{decrypt\_symmetric() (in module encryption)@\spxentry{decrypt\_symmetric()}\spxextra{in module encryption}}

\begin{fulllineitems}
\phantomsection\label{\detokenize{encryption:encryption.decrypt_symmetric}}
\pysigstartsignatures
\pysiglinewithargsret{\sphinxcode{\sphinxupquote{encryption.}}\sphinxbfcode{\sphinxupquote{decrypt\_symmetric}}}{\sphinxparam{\DUrole{n}{encrypted\_data}}\sphinxparamcomma \sphinxparam{\DUrole{n}{key}}}{}
\pysigstopsignatures
\sphinxAtStartPar
Decrypts symmetrically encrypted data using AES decryption with a given key.
\begin{quote}\begin{description}
\sphinxlineitem{Parameters}\begin{itemize}
\item {} 
\sphinxAtStartPar
\sphinxstyleliteralstrong{\sphinxupquote{encrypted\_data}} (\sphinxstyleliteralemphasis{\sphinxupquote{bytes}}) \textendash{} The encrypted data.

\item {} 
\sphinxAtStartPar
\sphinxstyleliteralstrong{\sphinxupquote{key}} (\sphinxstyleliteralemphasis{\sphinxupquote{bytes}}) \textendash{} The symmetric decryption key.

\end{itemize}

\sphinxlineitem{Returns}
\sphinxAtStartPar
The decrypted data.

\sphinxlineitem{Return type}
\sphinxAtStartPar
bytes

\end{description}\end{quote}

\end{fulllineitems}

\index{encrypt\_rsa() (in module encryption)@\spxentry{encrypt\_rsa()}\spxextra{in module encryption}}

\begin{fulllineitems}
\phantomsection\label{\detokenize{encryption:encryption.encrypt_rsa}}
\pysigstartsignatures
\pysiglinewithargsret{\sphinxcode{\sphinxupquote{encryption.}}\sphinxbfcode{\sphinxupquote{encrypt\_rsa}}}{\sphinxparam{\DUrole{n}{message}}\sphinxparamcomma \sphinxparam{\DUrole{n}{public\_key}}}{}
\pysigstopsignatures
\sphinxAtStartPar
Encrypts data using RSA encryption with a given public key.
\begin{quote}\begin{description}
\sphinxlineitem{Parameters}\begin{itemize}
\item {} 
\sphinxAtStartPar
\sphinxstyleliteralstrong{\sphinxupquote{message}} (\sphinxstyleliteralemphasis{\sphinxupquote{bytes}}) \textendash{} The data to encrypt.

\item {} 
\sphinxAtStartPar
\sphinxstyleliteralstrong{\sphinxupquote{public\_key}} (\sphinxstyleliteralemphasis{\sphinxupquote{RSA.RsaKey}}) \textendash{} The RSA public key.

\end{itemize}

\sphinxlineitem{Returns}
\sphinxAtStartPar
The encrypted data.

\sphinxlineitem{Return type}
\sphinxAtStartPar
bytes

\end{description}\end{quote}

\end{fulllineitems}

\index{encrypt\_symmetric() (in module encryption)@\spxentry{encrypt\_symmetric()}\spxextra{in module encryption}}

\begin{fulllineitems}
\phantomsection\label{\detokenize{encryption:encryption.encrypt_symmetric}}
\pysigstartsignatures
\pysiglinewithargsret{\sphinxcode{\sphinxupquote{encryption.}}\sphinxbfcode{\sphinxupquote{encrypt\_symmetric}}}{\sphinxparam{\DUrole{n}{data}}\sphinxparamcomma \sphinxparam{\DUrole{n}{key}}}{}
\pysigstopsignatures
\sphinxAtStartPar
Encrypts data symmetrically using AES encryption with a given key.
\begin{quote}\begin{description}
\sphinxlineitem{Parameters}\begin{itemize}
\item {} 
\sphinxAtStartPar
\sphinxstyleliteralstrong{\sphinxupquote{data}} (\sphinxstyleliteralemphasis{\sphinxupquote{bytes}}) \textendash{} The data to encrypt.

\item {} 
\sphinxAtStartPar
\sphinxstyleliteralstrong{\sphinxupquote{key}} (\sphinxstyleliteralemphasis{\sphinxupquote{bytes}}) \textendash{} The symmetric encryption key.

\end{itemize}

\sphinxlineitem{Returns}
\sphinxAtStartPar
The encrypted data.

\sphinxlineitem{Return type}
\sphinxAtStartPar
bytes

\end{description}\end{quote}

\end{fulllineitems}

\index{generate\_key\_pair() (in module encryption)@\spxentry{generate\_key\_pair()}\spxextra{in module encryption}}

\begin{fulllineitems}
\phantomsection\label{\detokenize{encryption:encryption.generate_key_pair}}
\pysigstartsignatures
\pysiglinewithargsret{\sphinxcode{\sphinxupquote{encryption.}}\sphinxbfcode{\sphinxupquote{generate\_key\_pair}}}{}{}
\pysigstopsignatures
\sphinxAtStartPar
Generates a pair of RSA public and private keys.
\begin{quote}\begin{description}
\sphinxlineitem{Returns}
\sphinxAtStartPar
The RSA public and private keys.

\sphinxlineitem{Return type}
\sphinxAtStartPar
tuple

\end{description}\end{quote}

\end{fulllineitems}


\sphinxstepscope


\section{get\_log\_messages module}
\label{\detokenize{get_log_messages:module-get_log_messages}}\label{\detokenize{get_log_messages:get-log-messages-module}}\label{\detokenize{get_log_messages::doc}}\index{module@\spxentry{module}!get\_log\_messages@\spxentry{get\_log\_messages}}\index{get\_log\_messages@\spxentry{get\_log\_messages}!module@\spxentry{module}}\index{decrypt\_log\_messages() (in module get\_log\_messages)@\spxentry{decrypt\_log\_messages()}\spxextra{in module get\_log\_messages}}

\begin{fulllineitems}
\phantomsection\label{\detokenize{get_log_messages:get_log_messages.decrypt_log_messages}}
\pysigstartsignatures
\pysiglinewithargsret{\sphinxcode{\sphinxupquote{get\_log\_messages.}}\sphinxbfcode{\sphinxupquote{decrypt\_log\_messages}}}{}{}
\pysigstopsignatures
\sphinxAtStartPar
Decrypts log messages stored in a SQLite database using an encrypted private key.

\end{fulllineitems}

\index{decrypt\_private\_key() (in module get\_log\_messages)@\spxentry{decrypt\_private\_key()}\spxextra{in module get\_log\_messages}}

\begin{fulllineitems}
\phantomsection\label{\detokenize{get_log_messages:get_log_messages.decrypt_private_key}}
\pysigstartsignatures
\pysiglinewithargsret{\sphinxcode{\sphinxupquote{get\_log\_messages.}}\sphinxbfcode{\sphinxupquote{decrypt\_private\_key}}}{\sphinxparam{\DUrole{n}{encrypted\_key\_data}}\sphinxparamcomma \sphinxparam{\DUrole{n}{passphrase}}}{}
\pysigstopsignatures
\sphinxAtStartPar
Decrypts the private key using the provided passphrase.
\begin{description}
\sphinxlineitem{Args:}
\sphinxAtStartPar
encrypted\_key\_data (bytes): The encrypted private key data.
passphrase (str): The passphrase used for decryption.

\sphinxlineitem{Returns:}
\sphinxAtStartPar
RSA.RsaKey or None: The decrypted private key if successful, None otherwise.

\end{description}

\end{fulllineitems}

\index{get\_private\_key\_password() (in module get\_log\_messages)@\spxentry{get\_private\_key\_password()}\spxextra{in module get\_log\_messages}}

\begin{fulllineitems}
\phantomsection\label{\detokenize{get_log_messages:get_log_messages.get_private_key_password}}
\pysigstartsignatures
\pysiglinewithargsret{\sphinxcode{\sphinxupquote{get\_log\_messages.}}\sphinxbfcode{\sphinxupquote{get\_private\_key\_password}}}{\sphinxparam{\DUrole{n}{encrypted\_private\_key}}}{}
\pysigstopsignatures
\sphinxAtStartPar
Prompts the user for the private key password and attempts to decrypt the private key.
\begin{description}
\sphinxlineitem{Args:}
\sphinxAtStartPar
encrypted\_private\_key (bytes): The encrypted private key data.

\sphinxlineitem{Returns:}
\sphinxAtStartPar
RSA.RsaKey: The decrypted private key if successful.

\end{description}

\end{fulllineitems}


\sphinxstepscope


\section{integrity module}
\label{\detokenize{integrity:module-integrity}}\label{\detokenize{integrity:integrity-module}}\label{\detokenize{integrity::doc}}\index{module@\spxentry{module}!integrity@\spxentry{integrity}}\index{integrity@\spxentry{integrity}!module@\spxentry{module}}\index{generate\_digest() (in module integrity)@\spxentry{generate\_digest()}\spxextra{in module integrity}}

\begin{fulllineitems}
\phantomsection\label{\detokenize{integrity:integrity.generate_digest}}
\pysigstartsignatures
\pysiglinewithargsret{\sphinxcode{\sphinxupquote{integrity.}}\sphinxbfcode{\sphinxupquote{generate\_digest}}}{\sphinxparam{\DUrole{n}{message}}\sphinxparamcomma \sphinxparam{\DUrole{n}{key}}\sphinxparamcomma \sphinxparam{\DUrole{n}{mac\_algorithm}}}{}
\pysigstopsignatures
\sphinxAtStartPar
Generate a digest for the given message using the specified key and MAC algorithm.
\begin{description}
\sphinxlineitem{Args:}
\sphinxAtStartPar
message (bytes or str): The message to generate the digest for.
key (bytes or str): The key used for generating the digest.
mac\_algorithm (str): The MAC algorithm to use for hashing.

\sphinxlineitem{Returns:}
\sphinxAtStartPar
bytes: The generated digest.

\end{description}

\end{fulllineitems}

\index{verify\_digest() (in module integrity)@\spxentry{verify\_digest()}\spxextra{in module integrity}}

\begin{fulllineitems}
\phantomsection\label{\detokenize{integrity:integrity.verify_digest}}
\pysigstartsignatures
\pysiglinewithargsret{\sphinxcode{\sphinxupquote{integrity.}}\sphinxbfcode{\sphinxupquote{verify\_digest}}}{\sphinxparam{\DUrole{n}{digest}}\sphinxparamcomma \sphinxparam{\DUrole{n}{computed\_digest}}}{}
\pysigstopsignatures
\sphinxAtStartPar
Verify the integrity of a digest.
\begin{description}
\sphinxlineitem{Args:}
\sphinxAtStartPar
digest (bytes): The original digest to compare against.
computed\_digest (bytes): The computed digest to compare with the original.

\sphinxlineitem{Returns:}
\sphinxAtStartPar
bool: True if the digests match, False otherwise.

\end{description}

\end{fulllineitems}


\sphinxstepscope


\section{main module}
\label{\detokenize{main:module-main}}\label{\detokenize{main:main-module}}\label{\detokenize{main::doc}}\index{module@\spxentry{module}!main@\spxentry{main}}\index{main@\spxentry{main}!module@\spxentry{module}}\index{clear\_screen() (in module main)@\spxentry{clear\_screen()}\spxextra{in module main}}

\begin{fulllineitems}
\phantomsection\label{\detokenize{main:main.clear_screen}}
\pysigstartsignatures
\pysiglinewithargsret{\sphinxcode{\sphinxupquote{main.}}\sphinxbfcode{\sphinxupquote{clear\_screen}}}{}{}
\pysigstopsignatures
\sphinxAtStartPar
Clear the screen based on the operating system.

\end{fulllineitems}

\index{encrypted\_chat\_menu() (in module main)@\spxentry{encrypted\_chat\_menu()}\spxextra{in module main}}

\begin{fulllineitems}
\phantomsection\label{\detokenize{main:main.encrypted_chat_menu}}
\pysigstartsignatures
\pysiglinewithargsret{\sphinxcode{\sphinxupquote{main.}}\sphinxbfcode{\sphinxupquote{encrypted\_chat\_menu}}}{}{}
\pysigstopsignatures
\sphinxAtStartPar
Display the menu for the encrypted chat options.

\end{fulllineitems}

\index{main() (in module main)@\spxentry{main()}\spxextra{in module main}}

\begin{fulllineitems}
\phantomsection\label{\detokenize{main:main.main}}
\pysigstartsignatures
\pysiglinewithargsret{\sphinxcode{\sphinxupquote{main.}}\sphinxbfcode{\sphinxupquote{main}}}{}{}
\pysigstopsignatures
\sphinxAtStartPar
Main function to run the security application.

\end{fulllineitems}

\index{menu() (in module main)@\spxentry{menu()}\spxextra{in module main}}

\begin{fulllineitems}
\phantomsection\label{\detokenize{main:main.menu}}
\pysigstartsignatures
\pysiglinewithargsret{\sphinxcode{\sphinxupquote{main.}}\sphinxbfcode{\sphinxupquote{menu}}}{}{}
\pysigstopsignatures
\sphinxAtStartPar
Display the main menu options.

\end{fulllineitems}

\index{read\_log\_messages() (in module main)@\spxentry{read\_log\_messages()}\spxextra{in module main}}

\begin{fulllineitems}
\phantomsection\label{\detokenize{main:main.read_log_messages}}
\pysigstartsignatures
\pysiglinewithargsret{\sphinxcode{\sphinxupquote{main.}}\sphinxbfcode{\sphinxupquote{read\_log\_messages}}}{}{}
\pysigstopsignatures
\sphinxAtStartPar
Read the log messages for the encrypted chat.

\end{fulllineitems}

\index{start\_client() (in module main)@\spxentry{start\_client()}\spxextra{in module main}}

\begin{fulllineitems}
\phantomsection\label{\detokenize{main:main.start_client}}
\pysigstartsignatures
\pysiglinewithargsret{\sphinxcode{\sphinxupquote{main.}}\sphinxbfcode{\sphinxupquote{start\_client}}}{}{}
\pysigstopsignatures
\sphinxAtStartPar
Start the client for the encrypted chat.

\end{fulllineitems}

\index{start\_server() (in module main)@\spxentry{start\_server()}\spxextra{in module main}}

\begin{fulllineitems}
\phantomsection\label{\detokenize{main:main.start_server}}
\pysigstartsignatures
\pysiglinewithargsret{\sphinxcode{\sphinxupquote{main.}}\sphinxbfcode{\sphinxupquote{start\_server}}}{}{}
\pysigstopsignatures
\sphinxAtStartPar
Start the server for the encrypted chat.

\end{fulllineitems}


\sphinxstepscope


\section{portscanner module}
\label{\detokenize{portscanner:module-portscanner}}\label{\detokenize{portscanner:portscanner-module}}\label{\detokenize{portscanner::doc}}\index{module@\spxentry{module}!portscanner@\spxentry{portscanner}}\index{portscanner@\spxentry{portscanner}!module@\spxentry{module}}\index{generate\_csv() (in module portscanner)@\spxentry{generate\_csv()}\spxextra{in module portscanner}}

\begin{fulllineitems}
\phantomsection\label{\detokenize{portscanner:portscanner.generate_csv}}
\pysigstartsignatures
\pysiglinewithargsret{\sphinxcode{\sphinxupquote{portscanner.}}\sphinxbfcode{\sphinxupquote{generate\_csv}}}{\sphinxparam{\DUrole{n}{data}}\sphinxparamcomma \sphinxparam{\DUrole{n}{file\_path}}}{}
\pysigstopsignatures
\sphinxAtStartPar
Generate a CSV file containing the provided data.
\begin{quote}\begin{description}
\sphinxlineitem{Parameters}\begin{itemize}
\item {} 
\sphinxAtStartPar
\sphinxstyleliteralstrong{\sphinxupquote{data}} (\sphinxstyleliteralemphasis{\sphinxupquote{list}}\sphinxstyleliteralemphasis{\sphinxupquote{{[}}}\sphinxstyleliteralemphasis{\sphinxupquote{list}}\sphinxstyleliteralemphasis{\sphinxupquote{{[}}}\sphinxstyleliteralemphasis{\sphinxupquote{Any}}\sphinxstyleliteralemphasis{\sphinxupquote{{]}}}\sphinxstyleliteralemphasis{\sphinxupquote{{]}}}) \textendash{} The data to include in the CSV file.

\item {} 
\sphinxAtStartPar
\sphinxstyleliteralstrong{\sphinxupquote{file\_path}} (\sphinxstyleliteralemphasis{\sphinxupquote{str}}) \textendash{} The file path where the CSV file will be saved.

\end{itemize}

\end{description}\end{quote}

\end{fulllineitems}

\index{generate\_pdf() (in module portscanner)@\spxentry{generate\_pdf()}\spxextra{in module portscanner}}

\begin{fulllineitems}
\phantomsection\label{\detokenize{portscanner:portscanner.generate_pdf}}
\pysigstartsignatures
\pysiglinewithargsret{\sphinxcode{\sphinxupquote{portscanner.}}\sphinxbfcode{\sphinxupquote{generate\_pdf}}}{\sphinxparam{\DUrole{n}{report\_data}}\sphinxparamcomma \sphinxparam{\DUrole{n}{file\_path}}}{}
\pysigstopsignatures
\sphinxAtStartPar
Generate a PDF report containing the provided data.
\begin{quote}\begin{description}
\sphinxlineitem{Parameters}\begin{itemize}
\item {} 
\sphinxAtStartPar
\sphinxstyleliteralstrong{\sphinxupquote{report\_data}} (\sphinxstyleliteralemphasis{\sphinxupquote{list}}\sphinxstyleliteralemphasis{\sphinxupquote{{[}}}\sphinxstyleliteralemphasis{\sphinxupquote{str}}\sphinxstyleliteralemphasis{\sphinxupquote{{]}}}) \textendash{} The data to include in the PDF report.

\item {} 
\sphinxAtStartPar
\sphinxstyleliteralstrong{\sphinxupquote{file\_path}} (\sphinxstyleliteralemphasis{\sphinxupquote{str}}) \textendash{} The file path where the PDF report will be saved.

\end{itemize}

\end{description}\end{quote}

\end{fulllineitems}

\index{get\_port\_description() (in module portscanner)@\spxentry{get\_port\_description()}\spxextra{in module portscanner}}

\begin{fulllineitems}
\phantomsection\label{\detokenize{portscanner:portscanner.get_port_description}}
\pysigstartsignatures
\pysiglinewithargsret{\sphinxcode{\sphinxupquote{portscanner.}}\sphinxbfcode{\sphinxupquote{get\_port\_description}}}{\sphinxparam{\DUrole{n}{port}}}{}
\pysigstopsignatures
\sphinxAtStartPar
Retrieve the description of a port.
\begin{quote}\begin{description}
\sphinxlineitem{Parameters}
\sphinxAtStartPar
\sphinxstyleliteralstrong{\sphinxupquote{port}} (\sphinxstyleliteralemphasis{\sphinxupquote{int}}) \textendash{} The port number.

\sphinxlineitem{Returns}
\sphinxAtStartPar
The description of the port.

\sphinxlineitem{Return type}
\sphinxAtStartPar
str

\end{description}\end{quote}

\end{fulllineitems}

\index{scantcp() (in module portscanner)@\spxentry{scantcp()}\spxextra{in module portscanner}}

\begin{fulllineitems}
\phantomsection\label{\detokenize{portscanner:portscanner.scantcp}}
\pysigstartsignatures
\pysiglinewithargsret{\sphinxcode{\sphinxupquote{portscanner.}}\sphinxbfcode{\sphinxupquote{scantcp}}}{\sphinxparam{\DUrole{n}{r1}}\sphinxparamcomma \sphinxparam{\DUrole{n}{r2}}}{}
\pysigstopsignatures
\sphinxAtStartPar
Scan TCP ports within the specified range.
\begin{quote}\begin{description}
\sphinxlineitem{Parameters}\begin{itemize}
\item {} 
\sphinxAtStartPar
\sphinxstyleliteralstrong{\sphinxupquote{r1}} (\sphinxstyleliteralemphasis{\sphinxupquote{int}}) \textendash{} The starting port number.

\item {} 
\sphinxAtStartPar
\sphinxstyleliteralstrong{\sphinxupquote{r2}} (\sphinxstyleliteralemphasis{\sphinxupquote{int}}) \textendash{} The ending port number.

\end{itemize}

\end{description}\end{quote}

\end{fulllineitems}


\sphinxstepscope


\section{server1 module}
\label{\detokenize{server1:module-server1}}\label{\detokenize{server1:server1-module}}\label{\detokenize{server1::doc}}\index{module@\spxentry{module}!server1@\spxentry{server1}}\index{server1@\spxentry{server1}!module@\spxentry{module}}\index{Server (class in server1)@\spxentry{Server}\spxextra{class in server1}}

\begin{fulllineitems}
\phantomsection\label{\detokenize{server1:server1.Server}}
\pysigstartsignatures
\pysiglinewithargsret{\sphinxbfcode{\sphinxupquote{class\DUrole{w}{ }}}\sphinxcode{\sphinxupquote{server1.}}\sphinxbfcode{\sphinxupquote{Server}}}{\sphinxparam{\DUrole{n}{host}}\sphinxparamcomma \sphinxparam{\DUrole{n}{port}}}{}
\pysigstopsignatures
\sphinxAtStartPar
Bases: \sphinxcode{\sphinxupquote{object}}
\index{broadcast() (server1.Server method)@\spxentry{broadcast()}\spxextra{server1.Server method}}

\begin{fulllineitems}
\phantomsection\label{\detokenize{server1:server1.Server.broadcast}}
\pysigstartsignatures
\pysiglinewithargsret{\sphinxbfcode{\sphinxupquote{broadcast}}}{\sphinxparam{\DUrole{n}{message}}\sphinxparamcomma \sphinxparam{\DUrole{n}{sender\_username}}}{}
\pysigstopsignatures
\sphinxAtStartPar
Broadcasts a message to all users except the sender.
\begin{quote}\begin{description}
\sphinxlineitem{Parameters}\begin{itemize}
\item {} 
\sphinxAtStartPar
\sphinxstyleliteralstrong{\sphinxupquote{message}} (\sphinxstyleliteralemphasis{\sphinxupquote{str}}) \textendash{} The message to broadcast.

\item {} 
\sphinxAtStartPar
\sphinxstyleliteralstrong{\sphinxupquote{sender\_username}} (\sphinxstyleliteralemphasis{\sphinxupquote{str}}) \textendash{} The username of the sender.

\end{itemize}

\end{description}\end{quote}

\end{fulllineitems}

\index{encrypt\_file() (server1.Server method)@\spxentry{encrypt\_file()}\spxextra{server1.Server method}}

\begin{fulllineitems}
\phantomsection\label{\detokenize{server1:server1.Server.encrypt_file}}
\pysigstartsignatures
\pysiglinewithargsret{\sphinxbfcode{\sphinxupquote{encrypt\_file}}}{\sphinxparam{\DUrole{n}{plaintext}}\sphinxparamcomma \sphinxparam{\DUrole{n}{public\_key}}\sphinxparamcomma \sphinxparam{\DUrole{n}{username}}}{}
\pysigstopsignatures
\sphinxAtStartPar
Encrypts a plaintext file.
\begin{quote}\begin{description}
\sphinxlineitem{Parameters}\begin{itemize}
\item {} 
\sphinxAtStartPar
\sphinxstyleliteralstrong{\sphinxupquote{plaintext}} (\sphinxstyleliteralemphasis{\sphinxupquote{str}}) \textendash{} The plaintext to encrypt.

\item {} 
\sphinxAtStartPar
\sphinxstyleliteralstrong{\sphinxupquote{public\_key}} (\sphinxstyleliteralemphasis{\sphinxupquote{RSA.RsaKey}}) \textendash{} The public key for encryption.

\item {} 
\sphinxAtStartPar
\sphinxstyleliteralstrong{\sphinxupquote{username}} (\sphinxstyleliteralemphasis{\sphinxupquote{str}}) \textendash{} The username associated with the plaintext.

\end{itemize}

\end{description}\end{quote}

\end{fulllineitems}

\index{generate\_user\_mac\_key() (server1.Server method)@\spxentry{generate\_user\_mac\_key()}\spxextra{server1.Server method}}

\begin{fulllineitems}
\phantomsection\label{\detokenize{server1:server1.Server.generate_user_mac_key}}
\pysigstartsignatures
\pysiglinewithargsret{\sphinxbfcode{\sphinxupquote{generate\_user\_mac\_key}}}{}{}
\pysigstopsignatures
\sphinxAtStartPar
Generates a random MAC key for a user.
\begin{quote}\begin{description}
\sphinxlineitem{Returns}
\sphinxAtStartPar
The generated MAC key.

\sphinxlineitem{Return type}
\sphinxAtStartPar
bytes

\end{description}\end{quote}

\end{fulllineitems}

\index{handle\_client() (server1.Server method)@\spxentry{handle\_client()}\spxextra{server1.Server method}}

\begin{fulllineitems}
\phantomsection\label{\detokenize{server1:server1.Server.handle_client}}
\pysigstartsignatures
\pysiglinewithargsret{\sphinxbfcode{\sphinxupquote{handle\_client}}}{\sphinxparam{\DUrole{n}{client\_socket}}\sphinxparamcomma \sphinxparam{\DUrole{n}{client\_address}}}{}
\pysigstopsignatures
\sphinxAtStartPar
Handles communication with a client.
\begin{quote}\begin{description}
\sphinxlineitem{Parameters}\begin{itemize}
\item {} 
\sphinxAtStartPar
\sphinxstyleliteralstrong{\sphinxupquote{client\_socket}} (\sphinxstyleliteralemphasis{\sphinxupquote{socket.socket}}) \textendash{} The client socket.

\item {} 
\sphinxAtStartPar
\sphinxstyleliteralstrong{\sphinxupquote{client\_address}} (\sphinxstyleliteralemphasis{\sphinxupquote{tuple}}) \textendash{} The client’s address.

\end{itemize}

\end{description}\end{quote}

\end{fulllineitems}

\index{register\_user() (server1.Server method)@\spxentry{register\_user()}\spxextra{server1.Server method}}

\begin{fulllineitems}
\phantomsection\label{\detokenize{server1:server1.Server.register_user}}
\pysigstartsignatures
\pysiglinewithargsret{\sphinxbfcode{\sphinxupquote{register\_user}}}{\sphinxparam{\DUrole{n}{username}}\sphinxparamcomma \sphinxparam{\DUrole{n}{client\_socket}}\sphinxparamcomma \sphinxparam{\DUrole{n}{client\_public\_key}}}{}
\pysigstopsignatures
\sphinxAtStartPar
Registers a user with their username, socket, and public key.
\begin{quote}\begin{description}
\sphinxlineitem{Parameters}\begin{itemize}
\item {} 
\sphinxAtStartPar
\sphinxstyleliteralstrong{\sphinxupquote{username}} (\sphinxstyleliteralemphasis{\sphinxupquote{str}}) \textendash{} The username of the user.

\item {} 
\sphinxAtStartPar
\sphinxstyleliteralstrong{\sphinxupquote{client\_socket}} (\sphinxstyleliteralemphasis{\sphinxupquote{socket.socket}}) \textendash{} The socket object associated with the user.

\item {} 
\sphinxAtStartPar
\sphinxstyleliteralstrong{\sphinxupquote{client\_public\_key}} (\sphinxstyleliteralemphasis{\sphinxupquote{RSA.RsaKey}}) \textendash{} The public key of the user.

\end{itemize}

\end{description}\end{quote}

\end{fulllineitems}

\index{remove\_log\_messages() (server1.Server method)@\spxentry{remove\_log\_messages()}\spxextra{server1.Server method}}

\begin{fulllineitems}
\phantomsection\label{\detokenize{server1:server1.Server.remove_log_messages}}
\pysigstartsignatures
\pysiglinewithargsret{\sphinxbfcode{\sphinxupquote{remove\_log\_messages}}}{\sphinxparam{\DUrole{n}{client\_socket}}\sphinxparamcomma \sphinxparam{\DUrole{n}{sender\_username}}}{}
\pysigstopsignatures
\sphinxAtStartPar
Removes log messages for a user.
\begin{quote}\begin{description}
\sphinxlineitem{Parameters}\begin{itemize}
\item {} 
\sphinxAtStartPar
\sphinxstyleliteralstrong{\sphinxupquote{client\_socket}} (\sphinxstyleliteralemphasis{\sphinxupquote{socket.socket}}) \textendash{} The client socket.

\item {} 
\sphinxAtStartPar
\sphinxstyleliteralstrong{\sphinxupquote{sender\_username}} (\sphinxstyleliteralemphasis{\sphinxupquote{str}}) \textendash{} The username of the sender.

\end{itemize}

\end{description}\end{quote}

\end{fulllineitems}

\index{send\_log\_messages() (server1.Server method)@\spxentry{send\_log\_messages()}\spxextra{server1.Server method}}

\begin{fulllineitems}
\phantomsection\label{\detokenize{server1:server1.Server.send_log_messages}}
\pysigstartsignatures
\pysiglinewithargsret{\sphinxbfcode{\sphinxupquote{send\_log\_messages}}}{\sphinxparam{\DUrole{n}{client\_socket}}\sphinxparamcomma \sphinxparam{\DUrole{n}{sender\_username}}}{}
\pysigstopsignatures
\sphinxAtStartPar
Sends log messages to a client.
\begin{quote}\begin{description}
\sphinxlineitem{Parameters}\begin{itemize}
\item {} 
\sphinxAtStartPar
\sphinxstyleliteralstrong{\sphinxupquote{client\_socket}} (\sphinxstyleliteralemphasis{\sphinxupquote{socket.socket}}) \textendash{} The client socket.

\item {} 
\sphinxAtStartPar
\sphinxstyleliteralstrong{\sphinxupquote{sender\_username}} (\sphinxstyleliteralemphasis{\sphinxupquote{str}}) \textendash{} The username of the sender.

\end{itemize}

\end{description}\end{quote}

\end{fulllineitems}

\index{send\_mac\_key\_encrypted() (server1.Server method)@\spxentry{send\_mac\_key\_encrypted()}\spxextra{server1.Server method}}

\begin{fulllineitems}
\phantomsection\label{\detokenize{server1:server1.Server.send_mac_key_encrypted}}
\pysigstartsignatures
\pysiglinewithargsret{\sphinxbfcode{\sphinxupquote{send\_mac\_key\_encrypted}}}{\sphinxparam{\DUrole{n}{client\_socket}}\sphinxparamcomma \sphinxparam{\DUrole{n}{client\_public\_key}}}{}
\pysigstopsignatures
\sphinxAtStartPar
Sends the MAC key encrypted to a client.
\begin{quote}\begin{description}
\sphinxlineitem{Parameters}\begin{itemize}
\item {} 
\sphinxAtStartPar
\sphinxstyleliteralstrong{\sphinxupquote{client\_socket}} (\sphinxstyleliteralemphasis{\sphinxupquote{socket.socket}}) \textendash{} The client socket.

\item {} 
\sphinxAtStartPar
\sphinxstyleliteralstrong{\sphinxupquote{client\_public\_key}} (\sphinxstyleliteralemphasis{\sphinxupquote{RSA.RsaKey}}) \textendash{} The client’s public key.

\end{itemize}

\end{description}\end{quote}

\end{fulllineitems}

\index{send\_message() (server1.Server method)@\spxentry{send\_message()}\spxextra{server1.Server method}}

\begin{fulllineitems}
\phantomsection\label{\detokenize{server1:server1.Server.send_message}}
\pysigstartsignatures
\pysiglinewithargsret{\sphinxbfcode{\sphinxupquote{send\_message}}}{\sphinxparam{\DUrole{n}{client\_socket}}\sphinxparamcomma \sphinxparam{\DUrole{n}{message}}\sphinxparamcomma \sphinxparam{\DUrole{n}{sender\_username}}}{}
\pysigstopsignatures
\sphinxAtStartPar
Sends a message to a client.
\begin{quote}\begin{description}
\sphinxlineitem{Parameters}\begin{itemize}
\item {} 
\sphinxAtStartPar
\sphinxstyleliteralstrong{\sphinxupquote{client\_socket}} (\sphinxstyleliteralemphasis{\sphinxupquote{socket.socket}}) \textendash{} The client socket.

\item {} 
\sphinxAtStartPar
\sphinxstyleliteralstrong{\sphinxupquote{message}} (\sphinxstyleliteralemphasis{\sphinxupquote{str}}) \textendash{} The message to send.

\item {} 
\sphinxAtStartPar
\sphinxstyleliteralstrong{\sphinxupquote{sender\_username}} (\sphinxstyleliteralemphasis{\sphinxupquote{str}}) \textendash{} The username of the sender.

\end{itemize}

\end{description}\end{quote}

\end{fulllineitems}

\index{send\_public\_key() (server1.Server method)@\spxentry{send\_public\_key()}\spxextra{server1.Server method}}

\begin{fulllineitems}
\phantomsection\label{\detokenize{server1:server1.Server.send_public_key}}
\pysigstartsignatures
\pysiglinewithargsret{\sphinxbfcode{\sphinxupquote{send\_public\_key}}}{\sphinxparam{\DUrole{n}{client\_socket}}}{}
\pysigstopsignatures
\sphinxAtStartPar
Sends the server’s public key to a client.
\begin{quote}\begin{description}
\sphinxlineitem{Parameters}
\sphinxAtStartPar
\sphinxstyleliteralstrong{\sphinxupquote{client\_socket}} (\sphinxstyleliteralemphasis{\sphinxupquote{socket.socket}}) \textendash{} The client socket.

\end{description}\end{quote}

\end{fulllineitems}

\index{setup\_database() (server1.Server method)@\spxentry{setup\_database()}\spxextra{server1.Server method}}

\begin{fulllineitems}
\phantomsection\label{\detokenize{server1:server1.Server.setup_database}}
\pysigstartsignatures
\pysiglinewithargsret{\sphinxbfcode{\sphinxupquote{setup\_database}}}{}{}
\pysigstopsignatures
\sphinxAtStartPar
Sets up the SQLite database for storing messages.

\end{fulllineitems}

\index{start() (server1.Server method)@\spxentry{start()}\spxextra{server1.Server method}}

\begin{fulllineitems}
\phantomsection\label{\detokenize{server1:server1.Server.start}}
\pysigstartsignatures
\pysiglinewithargsret{\sphinxbfcode{\sphinxupquote{start}}}{}{}
\pysigstopsignatures
\sphinxAtStartPar
Starts the server and listens for incoming connections.

\end{fulllineitems}

\index{store\_message() (server1.Server method)@\spxentry{store\_message()}\spxextra{server1.Server method}}

\begin{fulllineitems}
\phantomsection\label{\detokenize{server1:server1.Server.store_message}}
\pysigstartsignatures
\pysiglinewithargsret{\sphinxbfcode{\sphinxupquote{store\_message}}}{\sphinxparam{\DUrole{n}{sender}}\sphinxparamcomma \sphinxparam{\DUrole{n}{message}}}{}
\pysigstopsignatures
\sphinxAtStartPar
Stores a message from a sender.
\begin{quote}\begin{description}
\sphinxlineitem{Parameters}\begin{itemize}
\item {} 
\sphinxAtStartPar
\sphinxstyleliteralstrong{\sphinxupquote{sender}} (\sphinxstyleliteralemphasis{\sphinxupquote{str}}) \textendash{} The username of the sender.

\item {} 
\sphinxAtStartPar
\sphinxstyleliteralstrong{\sphinxupquote{message}} (\sphinxstyleliteralemphasis{\sphinxupquote{str}}) \textendash{} The message to store.

\end{itemize}

\end{description}\end{quote}

\end{fulllineitems}


\end{fulllineitems}

\index{decrypt\_private\_key() (in module server1)@\spxentry{decrypt\_private\_key()}\spxextra{in module server1}}

\begin{fulllineitems}
\phantomsection\label{\detokenize{server1:server1.decrypt_private_key}}
\pysigstartsignatures
\pysiglinewithargsret{\sphinxcode{\sphinxupquote{server1.}}\sphinxbfcode{\sphinxupquote{decrypt\_private\_key}}}{\sphinxparam{\DUrole{n}{encrypted\_key\_data}}\sphinxparamcomma \sphinxparam{\DUrole{n}{passphrase}}}{}
\pysigstopsignatures
\sphinxAtStartPar
Decrypts an encrypted private key using a passphrase.
\begin{quote}\begin{description}
\sphinxlineitem{Parameters}\begin{itemize}
\item {} 
\sphinxAtStartPar
\sphinxstyleliteralstrong{\sphinxupquote{encrypted\_key\_data}} (\sphinxstyleliteralemphasis{\sphinxupquote{bytes}}) \textendash{} The encrypted private key data.

\item {} 
\sphinxAtStartPar
\sphinxstyleliteralstrong{\sphinxupquote{passphrase}} (\sphinxstyleliteralemphasis{\sphinxupquote{str}}) \textendash{} The passphrase used for decryption.

\end{itemize}

\sphinxlineitem{Returns}
\sphinxAtStartPar
The decrypted private key.

\sphinxlineitem{Return type}
\sphinxAtStartPar
RSA.RsaKey or None

\end{description}\end{quote}

\end{fulllineitems}

\index{encrypt\_private\_key() (in module server1)@\spxentry{encrypt\_private\_key()}\spxextra{in module server1}}

\begin{fulllineitems}
\phantomsection\label{\detokenize{server1:server1.encrypt_private_key}}
\pysigstartsignatures
\pysiglinewithargsret{\sphinxcode{\sphinxupquote{server1.}}\sphinxbfcode{\sphinxupquote{encrypt\_private\_key}}}{\sphinxparam{\DUrole{n}{private\_key}}\sphinxparamcomma \sphinxparam{\DUrole{n}{passphrase}}}{}
\pysigstopsignatures
\sphinxAtStartPar
Encrypts a private key with a passphrase.
\begin{quote}\begin{description}
\sphinxlineitem{Parameters}\begin{itemize}
\item {} 
\sphinxAtStartPar
\sphinxstyleliteralstrong{\sphinxupquote{private\_key}} (\sphinxstyleliteralemphasis{\sphinxupquote{RSA.RsaKey}}) \textendash{} The private key to be encrypted.

\item {} 
\sphinxAtStartPar
\sphinxstyleliteralstrong{\sphinxupquote{passphrase}} (\sphinxstyleliteralemphasis{\sphinxupquote{str}}) \textendash{} The passphrase to encrypt the private key.

\end{itemize}

\sphinxlineitem{Returns}
\sphinxAtStartPar
The encrypted private key.

\sphinxlineitem{Return type}
\sphinxAtStartPar
bytes

\end{description}\end{quote}

\end{fulllineitems}


\sphinxstepscope


\section{synflood module}
\label{\detokenize{synflood:module-synflood}}\label{\detokenize{synflood:synflood-module}}\label{\detokenize{synflood::doc}}\index{module@\spxentry{module}!synflood@\spxentry{synflood}}\index{synflood@\spxentry{synflood}!module@\spxentry{module}}\index{generate\_packet() (in module synflood)@\spxentry{generate\_packet()}\spxextra{in module synflood}}

\begin{fulllineitems}
\phantomsection\label{\detokenize{synflood:synflood.generate_packet}}
\pysigstartsignatures
\pysiglinewithargsret{\sphinxcode{\sphinxupquote{synflood.}}\sphinxbfcode{\sphinxupquote{generate\_packet}}}{\sphinxparam{\DUrole{n}{target\_ip}}}{}
\pysigstopsignatures
\sphinxAtStartPar
Generate a TCP SYN packet with specified destination IP.
\begin{quote}\begin{description}
\sphinxlineitem{Parameters}
\sphinxAtStartPar
\sphinxstyleliteralstrong{\sphinxupquote{target\_ip}} (\sphinxstyleliteralemphasis{\sphinxupquote{str}}) \textendash{} The destination IP address for the packet.

\sphinxlineitem{Returns}
\sphinxAtStartPar
The generated packet.

\sphinxlineitem{Return type}
\sphinxAtStartPar
scapy.Packet

\end{description}\end{quote}

\end{fulllineitems}

\index{send\_packets() (in module synflood)@\spxentry{send\_packets()}\spxextra{in module synflood}}

\begin{fulllineitems}
\phantomsection\label{\detokenize{synflood:synflood.send_packets}}
\pysigstartsignatures
\pysiglinewithargsret{\sphinxcode{\sphinxupquote{synflood.}}\sphinxbfcode{\sphinxupquote{send\_packets}}}{\sphinxparam{\DUrole{n}{target\_ip}}\sphinxparamcomma \sphinxparam{\DUrole{n}{num\_packets}}\sphinxparamcomma \sphinxparam{\DUrole{n}{num\_threads}}}{}
\pysigstopsignatures
\sphinxAtStartPar
Send packets to a target IP using multiple threads.
\begin{quote}\begin{description}
\sphinxlineitem{Parameters}\begin{itemize}
\item {} 
\sphinxAtStartPar
\sphinxstyleliteralstrong{\sphinxupquote{target\_ip}} (\sphinxstyleliteralemphasis{\sphinxupquote{str}}) \textendash{} The target IP address to send packets to.

\item {} 
\sphinxAtStartPar
\sphinxstyleliteralstrong{\sphinxupquote{num\_packets}} (\sphinxstyleliteralemphasis{\sphinxupquote{int}}) \textendash{} The number of packets to send per thread.

\item {} 
\sphinxAtStartPar
\sphinxstyleliteralstrong{\sphinxupquote{num\_threads}} (\sphinxstyleliteralemphasis{\sphinxupquote{int}}) \textendash{} The number of threads to use.

\end{itemize}

\end{description}\end{quote}

\end{fulllineitems}

\index{send\_packets\_thread() (in module synflood)@\spxentry{send\_packets\_thread()}\spxextra{in module synflood}}

\begin{fulllineitems}
\phantomsection\label{\detokenize{synflood:synflood.send_packets_thread}}
\pysigstartsignatures
\pysiglinewithargsret{\sphinxcode{\sphinxupquote{synflood.}}\sphinxbfcode{\sphinxupquote{send\_packets\_thread}}}{\sphinxparam{\DUrole{n}{target\_ip}}\sphinxparamcomma \sphinxparam{\DUrole{n}{num\_packets}}}{}
\pysigstopsignatures
\sphinxAtStartPar
Send a specified number of packets to a target IP in a separate thread.
\begin{quote}\begin{description}
\sphinxlineitem{Parameters}\begin{itemize}
\item {} 
\sphinxAtStartPar
\sphinxstyleliteralstrong{\sphinxupquote{target\_ip}} (\sphinxstyleliteralemphasis{\sphinxupquote{str}}) \textendash{} The target IP address to send packets to.

\item {} 
\sphinxAtStartPar
\sphinxstyleliteralstrong{\sphinxupquote{num\_packets}} (\sphinxstyleliteralemphasis{\sphinxupquote{int}}) \textendash{} The number of packets to send.

\end{itemize}

\end{description}\end{quote}

\end{fulllineitems}

\index{signal\_handler() (in module synflood)@\spxentry{signal\_handler()}\spxextra{in module synflood}}

\begin{fulllineitems}
\phantomsection\label{\detokenize{synflood:synflood.signal_handler}}
\pysigstartsignatures
\pysiglinewithargsret{\sphinxcode{\sphinxupquote{synflood.}}\sphinxbfcode{\sphinxupquote{signal\_handler}}}{\sphinxparam{\DUrole{n}{sig}}\sphinxparamcomma \sphinxparam{\DUrole{n}{frame}}}{}
\pysigstopsignatures
\sphinxAtStartPar
Handle KeyboardInterrupt signal.
\begin{quote}\begin{description}
\sphinxlineitem{Parameters}\begin{itemize}
\item {} 
\sphinxAtStartPar
\sphinxstyleliteralstrong{\sphinxupquote{sig}} (\sphinxstyleliteralemphasis{\sphinxupquote{int}}) \textendash{} The signal number.

\item {} 
\sphinxAtStartPar
\sphinxstyleliteralstrong{\sphinxupquote{frame}} (\sphinxstyleliteralemphasis{\sphinxupquote{frame object}}) \textendash{} The current stack frame.

\end{itemize}

\end{description}\end{quote}

\end{fulllineitems}


\sphinxstepscope


\section{udpflood module}
\label{\detokenize{udpflood:module-udpflood}}\label{\detokenize{udpflood:udpflood-module}}\label{\detokenize{udpflood::doc}}\index{module@\spxentry{module}!udpflood@\spxentry{udpflood}}\index{udpflood@\spxentry{udpflood}!module@\spxentry{module}}\index{send\_packet() (in module udpflood)@\spxentry{send\_packet()}\spxextra{in module udpflood}}

\begin{fulllineitems}
\phantomsection\label{\detokenize{udpflood:udpflood.send_packet}}
\pysigstartsignatures
\pysiglinewithargsret{\sphinxcode{\sphinxupquote{udpflood.}}\sphinxbfcode{\sphinxupquote{send\_packet}}}{\sphinxparam{\DUrole{n}{target\_ip}}\sphinxparamcomma \sphinxparam{\DUrole{n}{target\_port}}\sphinxparamcomma \sphinxparam{\DUrole{n}{msg\_to\_send}}}{}
\pysigstopsignatures
\sphinxAtStartPar
Function to send a UDP packet to a specified target IP and port with a given message.
\begin{quote}\begin{description}
\sphinxlineitem{Parameters}\begin{itemize}
\item {} 
\sphinxAtStartPar
\sphinxstyleliteralstrong{\sphinxupquote{target\_ip}} (\sphinxstyleliteralemphasis{\sphinxupquote{str}}) \textendash{} The target IP address to send the packet to.

\item {} 
\sphinxAtStartPar
\sphinxstyleliteralstrong{\sphinxupquote{target\_port}} (\sphinxstyleliteralemphasis{\sphinxupquote{int}}) \textendash{} The target port to send the packet to.

\item {} 
\sphinxAtStartPar
\sphinxstyleliteralstrong{\sphinxupquote{msg\_to\_send}} (\sphinxstyleliteralemphasis{\sphinxupquote{str}}) \textendash{} The message to be sent in the packet.

\end{itemize}

\end{description}\end{quote}

\end{fulllineitems}



\chapter{Indices and tables}
\label{\detokenize{index:indices-and-tables}}\begin{itemize}
\item {} 
\sphinxAtStartPar
\DUrole{xref,std,std-ref}{genindex}

\item {} 
\sphinxAtStartPar
\DUrole{xref,std,std-ref}{modindex}

\item {} 
\sphinxAtStartPar
\DUrole{xref,std,std-ref}{search}

\end{itemize}


\renewcommand{\indexname}{Python Module Index}
\begin{sphinxtheindex}
\let\bigletter\sphinxstyleindexlettergroup
\bigletter{c}
\item\relax\sphinxstyleindexentry{client1}\sphinxstyleindexpageref{client1:\detokenize{module-client1}}
\indexspace
\bigletter{d}
\item\relax\sphinxstyleindexentry{doorknocking}\sphinxstyleindexpageref{doorknocking:\detokenize{module-doorknocking}}
\indexspace
\bigletter{e}
\item\relax\sphinxstyleindexentry{encryption}\sphinxstyleindexpageref{encryption:\detokenize{module-encryption}}
\indexspace
\bigletter{g}
\item\relax\sphinxstyleindexentry{get\_log\_messages}\sphinxstyleindexpageref{get_log_messages:\detokenize{module-get_log_messages}}
\indexspace
\bigletter{i}
\item\relax\sphinxstyleindexentry{integrity}\sphinxstyleindexpageref{integrity:\detokenize{module-integrity}}
\indexspace
\bigletter{m}
\item\relax\sphinxstyleindexentry{main}\sphinxstyleindexpageref{main:\detokenize{module-main}}
\indexspace
\bigletter{p}
\item\relax\sphinxstyleindexentry{portscanner}\sphinxstyleindexpageref{portscanner:\detokenize{module-portscanner}}
\indexspace
\bigletter{s}
\item\relax\sphinxstyleindexentry{server1}\sphinxstyleindexpageref{server1:\detokenize{module-server1}}
\item\relax\sphinxstyleindexentry{synflood}\sphinxstyleindexpageref{synflood:\detokenize{module-synflood}}
\indexspace
\bigletter{u}
\item\relax\sphinxstyleindexentry{udpflood}\sphinxstyleindexpageref{udpflood:\detokenize{module-udpflood}}
\end{sphinxtheindex}

\renewcommand{\indexname}{Index}
\printindex
\end{document}